\documentclass{beamer}

\usepackage{beamerthemesplit}
\usepackage{graphicx}
\usepackage{color, natbib, hyperref}

% define colors
\definecolor{jblue}  {RGB}{20,50,100}
\definecolor{ngreen} {RGB}{98,158,31}

%theme

\usetheme{boxes} 
%\usecolortheme{seahorse} 
\setbeamertemplate{items}[default] 
%\setbeamercovered{transparent}
\setbeamertemplate{blocks}[rounded][shadow=true] 
\setbeamertemplate{navigation symbols}{} 
% set the basic colors
\setbeamercolor{palette primary}   {fg=black,bg=white}
\setbeamercolor{palette secondary} {fg=black,bg=white}
\setbeamercolor{palette tertiary}  {bg=jblue,fg=white}
\setbeamercolor{palette quaternary}{fg=black,bg=white}
\setbeamercolor{structure}{fg=jblue}
\setbeamercolor{titlelike}         {bg=jblue,fg=white}
\setbeamercolor{frametitle}        {bg=jblue!10,fg=jblue}
\setbeamercolor{cboxb}{fg=black,bg=jblue}
\setbeamercolor{cboxr}{fg=black,bg=red}

% set colors for itemize/enumerate
\setbeamercolor{item}{fg=ngreen}
\setbeamercolor{item projected}{fg=white,bg=ngreen}

% set colors for blocks
\setbeamercolor{block title}{fg=ngreen,bg=white}
\setbeamercolor{block body}{fg=black,bg=white}

% set colors for alerted blocks (blocks with frame)
%\setbeamercolor{block alerted title}{fg=white,bg=jblue}
%\setbeamercolor{block alerted body}{fg=black,bg=jblue!10}
\setbeamercolor{block alerted title}{fg=white,bg=dblue!70} % Colors of the highlighted block titles
\setbeamercolor{block alerted body}{fg=black,bg=dblue!10} % Colors of the body of highlighted blocks

% set the fonts
\usefonttheme{professionalfonts}

\setbeamerfont{section in head/foot}{series=\bfseries}
\setbeamerfont{block title}{series=\bfseries}
\setbeamerfont{block alerted title}{series=\bfseries}
\setbeamerfont{frametitle}{series=\bfseries}
\setbeamerfont{frametitle}{size=\Large}
\setbeamerfont{block body}{series=\rmfamily}
\setbeamerfont{caption}{series=\rmfamily}
\setbeamerfont{headline}{series=\rmfamily}


% set some beamer theme options
\setbeamertemplate{title page}[default][colsep=-4bp,rounded=true]
\setbeamertemplate{sections/subsections in toc}[square]
\setbeamertemplate{items}[circle]
\setbeamertemplate{blocks}[width=0.0]
\beamertemplatenavigationsymbolsempty

% Making a DAG
\usepackage{tkz-graph}  
\usetikzlibrary{shapes.geometric}
\tikzstyle{VertexStyle} = [shape            = ellipse,
                               minimum width    = 6ex,%
                               draw]
 \tikzstyle{EdgeStyle}   = [->,>=stealth']      




\mode<presentation>

\title[Model-based matching]{Model-based matching for causal inference in observational studies}
\author{Kellie Ottoboni \\ with Philip B. Stark}
\institute[]{Department of Statistics, UC Berkeley}
\date{March 14, 2016}

\begin{document}

\frame{\titlepage}

\frame
{
  \frametitle{Observational Studies vs Experiments}
 \begin{center}
\begin{itemize}
\item \textbf{Problem:} Estimate the causal effect of a treatment on outcome of interest
\item In randomized experiments, treatment is assigned to individuals at random.
\item In observational studies, the way individuals select into treatment groups is unknown.
%\item \textbf{Confounding} occurs when a variable affects both treatment assignment and the outcome. It biases our estimates of the treatment effect.
\end{itemize}

\begin{figure}[h]
\begin{tikzpicture}
\SetGraphUnit{2} 
\Vertex{Outcome} \NOWE(Outcome){Treatment} \NOEA(Outcome){Confounder}
\Edges[color=red, label=?](Treatment, Outcome) \Edges(Confounder, Outcome) \pause \Edges(Confounder, Treatment) 
\end{tikzpicture}
\end{figure}
\end{center}
}


\frame
{
  \frametitle{Matching}
\begin{center}
\begin{itemize}
\item \textbf{Ideal:} group individuals by important confounders to estimate subgroup treatment effects and then average over subgroups
\item \textbf{Reality:} many covariates, perhaps continuous, make it difficult to stratify
\end{itemize}
\begin{tikzpicture}[scale = 0.5]
\foreach \x in{0,...,4}
{   \draw (0,\x ,4) -- (4,\x ,4);
    \draw (\x ,0,4) -- (\x ,4,4);
    \draw (4,\x ,4) -- (4,\x ,0);
    \draw (\x ,4,4) -- (\x ,4,0);
    \draw (4,0,\x ) -- (4,4,\x );
    \draw (0,4,\x ) -- (4,4,\x );
}
\end{tikzpicture}
\begin{itemize}
\item \textbf{Solution:} use a one-dimensional score to match or group individuals
\item  Stratify on $\hat{Y}$, the ``best'' prediction of the response based on all covariates except for the treatment
\end{itemize}

\end{center}
}

\frame
{
  \frametitle{Estimation}
\begin{center}
\begin{itemize}
\item Under standard assumptions, we can estimate the \textbf{average treatment effect} nonparametrically using the difference in average residuals, $Y - \hat{Y}$, between treated and controls
\end{itemize}

\begin{figure}[htbp]
\begin{center}
\includegraphics[width = \textwidth]{fig/estimates.png}

\end{center}
\end{figure}


\end{center}
}

%\frame
%{
%  \frametitle{Model-based Matching}
%\begin{center}
%\begin{itemize}
%\item Use stratified permutation test to test the \textbf{strong null hypothesis} of no treatment effect whatsoever
%\item Stratifying on $\hat{Y}$ allows us to detect non-constant and non-linear treatment effects
%\end{itemize}
%
%\begin{figure}[htbp]
%\begin{center}
%\includegraphics[width = 0.9\textwidth]{fig/power.png}
%\end{center}
%\end{figure}
%
%
%\end{center}
%}

\frame
{
  \frametitle{Hypothesis testing: example}
\begin{center}
\begin{itemize}
\item Test the \textbf{strong null hypothesis} of no relationship between salt consumption and country-level mortality rates
\item Control for other health predictors: look at residuals $Y - \hat{Y}$
\item Stratified permutation test requires no distributional assumptions
%\item Stratifying on $\hat{Y}$ allows us to detect non-constant and non-linear treatment effects
%\item Test for association between residuals $Y- \hat{Y}$ and treatment
\end{itemize}

\begin{figure}[htbp]
\begin{center}
\includegraphics[width = \textwidth]{fig/sodium_lifeexp.png}
\end{center}
\end{figure}


\end{center}
}

\frame
{
  \frametitle{Future Directions}
\begin{center}
\begin{itemize}
\item Do different test statistics give greater power when the treatment effect is nonlinear?
\item What is the optimal way to stratify?
\item How to quantify uncertainty -- standard errors and confidence intervals?
\end{itemize}
\end{center}
}



%
%\section{References}
%\begin{frame}
%\frametitle{References}
%\bibliographystyle{plainnat}
%\bibliography{refs}
%\itemize
%\end{frame}

\end{document}
